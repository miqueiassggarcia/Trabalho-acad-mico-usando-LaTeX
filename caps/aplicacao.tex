%\pagenumbering{arabic}

\section{Aplicações}

Python possui uma vasta gama de possibilidades para suas aplicações em projetos de Inteligência Artificial (IA), desde análise profunda de textos até automóveis autônomos, isso se faz possível pela sua grande quantidade de bibliotecas e também por sua simplificada metodologia de aplicação.\cite{didatica2024}

\subsection{Análise profunda de textos}

A análise profunda de textos, realizada através da mineração de textos, é uma aplicação comum de IA que utiliza \textit{Python}. Bibliotecas como NLTK e TextBlob permitem aos desenvolvedores analisar o tom emocional por trás de palavras e frases em grandes conjuntos de dados, abrindo portas para \textit{insights} em áreas como mercado de ações, avaliações de produtos e tendências de opinião pública possibilitando assim a extração de sentimentos dos textos.\cite{didatica2024}

\subsection{Automovéis Autônomos}

Python também desempenha um papel fundamental no desenvolvimento de tecnologias para veículos autônomos. Através de bibliotecas como OpenCV para processamento de imagens e \textit{TensorFlow} para aprendizado de máquina, desenvolvedores conseguem criar sistemas complexos de reconhecimento de padrões e tomada de decisão, essenciais para a navegação autônoma.\cite{didatica2024}