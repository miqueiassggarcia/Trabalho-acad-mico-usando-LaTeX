\chapter*{Introdução}

Vislumbrando a crescente ascensão da Inteligência Artificial no mundo moderno e analisando a grande quantidade de linguagens de programação, a seguinte pesquisa buscando uma linguagem mais simplificada porém com grande possibilidades de utilização optou pela utilização de \textit{Python}, assim uma grande mineração por conteúdos relacionados foi iniciada.

Então em uma breve explicação, o que seria \textit{Python} e o porque dessa linguagem ter sido escolhida como disse \cite{pythonbasico2024} \textit{Python} é uma linguagem de alto nível orientada a objetos, criado pelo Holandês Guido Van Rossum sob o ideal, " Programação de computadores para todos." Com isso temos a visão do porque \textit{Python} é uma boa opção, principalmente para uso em conjunto com a IA \cite{menezes2010introduccao} diz que ela é linguagem clara e objetiva, pois é direto ao ponto. \textit{Python} é software livre, ou seja, pode ser utilizada gratuitamente. possibilitando seu uso em praticamente qualquer arquitetura de computadores ou sistema operacional, como Linux, FreeBSD, Microsoft Windows ou Mac OS X.

E ainda acresenta que seu uso vem crescendo em várias áreas da computação, como inteligência artificial que é nosso foco de pesquisa em conjunto, entre outras áreas como, banco de dados, biotecnologia, animação 3D, aplicativos móveis (celulares), jogos e mesmo como plataforma web.

\newpage
\section{Objetivos}
Os objetivo principal do trabalho é apresentar a linguagem de programação \textit{Python} e demonstrar suas principais utilidades na área de Inteligência Artificial e sua principais aplicações nessa área que está em alta nós dias atuais.

\section{Estrutura do trabalho}
O trabalho será composto por os seguintes tópicos:
\begin{itemize}
   \item {Introdução}: Neste capítulo será introduzido sobre o tema principal do trabalho apresentando sobre o objetivo central e estrutura do trabalho.
   \item {Linguagem}: Neste capítulo será dada uma breve introdução da linguagem python e das suas principais características.
   \item {Principais Implementações}: Neste capítulo será apresentada algumas das principais possibilidades de aplicação que são utilizadas atualmente do \textit{Python} na área de Inteligência Artificial.
   \item {Versões}: Neste capítulo será descrito como funciona o versionamento no \textit{Python} e em seguida listada as principais versões utilizadas nós dias atuais.
   \item {Aplicação}: Neste capítulo será apresentado algumas das principais formas que o \textit{Python} é aplicado atualmente.
   \item {Conclusão}: Neste capítulo será feita a conclusão e finalização do trabalho, com as considerações finais do trabalho.
\end{itemize}
